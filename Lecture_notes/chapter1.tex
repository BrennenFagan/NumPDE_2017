%%%%%%%%%%%%%%%%%%%%%%%%%%%%%%%%%
%
% Lecture notes for Numerical Methods for Partial Differential Equations
%
% Chapter 1: Introduction
%
%%%%%%%%%%%%%%%%%%%%%%%%%%%%%%%%%

% !TeX root = NumPDE_Lecture_notes.tex

\section{Introduction} \label{s:1}

The overall aim of this module is to show how computers
can be used to solve various mathematical problems for partial
differential equations (PDEs). This involves both theoretical and
practical components. The theoretical part is an introduction to
the most commonly used numerical methods for solving PDEs. Here we
will discuss how and why these methods work. In the practical
part, we will use R and C++ to demonstrate how the numerical methods
discussed in the theoretical part can be implemented in practice.

We will discuss numerical methods for
parabolic (the heat equation), elliptic (the Laplace and Poisson
equations) and hyperbolic (the wave equation) PDEs. 

There are of course many good textbooks on the subject. Three that we
will refer to from time to time are:
\begin{enumerate}

\item{} RL Burden \& JD Faires, {\it Numerical Analysis}
(6th ed.), Brooks/Cole Publishing Company, 1997;

\item{} WF Ames,
{\it Numerical Methods for Partial Differential Equations}, Academic
Press, 1977;

\item{} WH Press,
{\it Numerical Recipes: the Art of Scientific Computing}, CUP, 2007.

\end{enumerate}

Below are some facts from Calculus that are used throughout this course.

\begin{theorem}[Taylor's theorem for functions of one variable]
Let $f\in C^{n+1}$ in the neighbourhood of the
point $x_{0}$ (i.e. $f$ is continuous and has continuous
derivatives of all orders up to the $(n+1)$th order). Then, for all
$x$ in this neighbourhood,
\[
f(x)=T_{n}+R_{n}
\]
where $T_{n}$ in the $n$th Taylor polynomial
\[
T_{n}=f(x_{0})+(x-x_{0})f^{\prime}(x_{0})+
\frac{(x-x_{0})^{2}}{2!}f^{\prime\prime}(x_{0})+ \dots+
\frac{(x-x_{0})^{n}}{n!}f^{(n)}(x_{0})
\]
and $R_{n}$ is the remainder term:
\[
R_{n}=\frac{(x-x_{0})^{n+1}}{(n+1)!}f^{(n+1)}(\xi)
\]
for some point $\xi$ between $x_{0}$ and $x$ 
%($\xi$ can be written
%as $\xi=x_{0}+\theta (x-x_{0})$ where $0< \theta<1$). 
\end{theorem}

\begin{example}
Let us obtain the Taylor series expansion of
$f(x)=\sin x$ about the point $x_{0}=0$.
We have
\begin{gather*}
f(0)=0, ~~~~ f^{\prime}(0)=\cos x\vert_{x=0}=1, ~~~~
f^{\prime\prime}(0)=-\sin x\vert_{x=0}=0, \\
f^{\prime\prime\prime}(0)=-\cos x\vert_{x=0}=-1, ~~~~
{\rm etc.}
\end{gather*}
Hence,
\[
\sin x=x-\frac{x^{3}}{3!}+\frac{x^{5}}{5!}-\dots=
\sum_{n=1}^{\infty}(-1)^{n-1}\frac{x^{2n-1}}{(2n-1)!}.
\]
If we restrict our attention to the $n$th Taylor polynomial for
$\sin x$, then the remainder term $R_{n}$ can be estimated using the
fact that $f^{(n+1)}(\xi)$ is equal to
either $\pm\sin x$ or $\pm\cos x$ depending on $n$. In both cases
$\vert f^{(n+1)}(\xi)\vert\leq 1$. Hence, we obtain
\[
\vert R_{n}\vert \leq\frac{\vert x-x_{0}\vert^{n+1}}{(n+1)!}.
\]
\end{example}

\begin{definition} Let $\lim\limits_{x\to 0}g(x)=0$ and
$\lim\limits_{x\to 0}f(x)=f_{0}$. If there exists
a positive constant $K$ such that
\[
\vert f(x)-f_{0}\vert   \leq K \vert g(x)\vert,
\]
at least when $x$ is sufficiently close to zero, we write
\[
f(x)=f_{0}+O(g(x))
\]
as $x\to 0$.
\end{definition}
Note that in the above definition it is important that we specify the ``as $x\to 0$''.
We are interested in the behaviour as $x$ gets smaller, and we say something is
$O(g(x))$ if it goes to zero at least as fast as $g(x)$.
One could also use the Big Oh notation for other limits, in particular $x\to\infty$.

\begin{example}
The function $f(x)=\sin(x)/x$ converges to 1 as fast
as $x^{2}$ converges to zero (as $x\to 0$). To show this,
it suffices to consider the second Taylor polynomial for $\sin(x)$:
\[
\sin(x)=x-{x^{3} \over 3!}\cos(\xi)
\]
where $\xi$ is some number between 0 and $x$. We have
\[
\left\vert \frac{\sin x}{x} -1 \right\vert = \frac{\vert x
\vert^{2}}{3!}\vert \cos(\xi)\vert \leq \frac{x^{2}}{3!}=\frac{x^{2}}{6} \ \ \
\Rightarrow \ \ \ \ \frac{\sin x}{x}=1+O(x^{2}).
\]
Here we used the fact that $\vert\cos(\xi)\vert\leq 1$ for all $\xi$.
\end{example}

\begin{lemma}[Properties of $O(x^n)$ as $x\to 0$]
We have
\begin{enumerate}

\item $O(x^n)+O(x^m)=O(x^l)$ \ \ for $n,m \geq 0$ \ \ and \ \ $l=\min\{n,m\}$.

\item $O(x^n)O(x^m)=O(x^{n+m})$ \ \ for \ \ $n,m \geq 0$.

\item $x^m O(x^n)=O(x^{n+m})$ \ \ for $n \geq 0$ \ \ and \ \ $n+m \geq 0$.
\end{enumerate}
\end{lemma}

\noindent
For example,
\[
O(x^2)+O(x^3)=O(x^2), \quad
O(x^2)O(x^3)=O(x^5), \quad
x^{-2}O(x^3)=O(x).
\]
Note that the first property holds for $x\to 0$ but if we were instead considering
$x\to\infty$ the min would change to a max.

\pagebreak[2]

\begin{theorem}[Taylor's theorem for functions of two variables]
Suppose that $f(x, y)$ and all its partial derivatives of order less
than or equal to $(n+1)$ are continuous in
$D=\{ (x, y) \ \vert \ a< x< b, \ c< y< d \}$, and let $(x_{0}, y_{0})\in D$.
For every $(x, y)\in D$, there exist $\xi$ between $x$ and $x_{0}$ and $\mu$
between $y$ and $y_{0}$ such that
$$
f(x, y)=T_{n}(x, y)+R_{n}(x, y)
$$
where
\begin{eqnarray}
T_{n}(x, y) &=& f(x_{0}, y_{0})+
\left[(x-x_{0})\frac{\partial f}{\partial x}(x_{0}, y_{0})+
(y-y_{0})\frac{\partial f}{\partial y}(x_{0}, y_{0})\right] \nonumber \\
&& + \left[\frac{(x-x_{0})^{2}}{2}\frac{\partial^{2} f}{\partial x^{2}}(x_{0}, y_{0})+
(x-x_{0})(y-y_{0})\frac{\partial^{2} f}{\partial x\partial y}(x_{0}, y_{0})\right.\\
&&\qquad\qquad\qquad+\left.
\frac{(y-y_{0})^{2}}{2}\frac{\partial^{2} f}{\partial y^{2}}(x_{0}, y_{0})
\right] + .... \nonumber \\
&& + \left[\frac{1}{n!}\sum_{j=0}^{n}
\left(
\begin{array}{c}
n \\
 j
\end{array}
\right)
(x-x_{0})^{n-j}(y-y_{0})^{j}
\frac{\partial^{n} f}{\partial x^{n-j}\partial y^{j}}(x_{0}, y_{0})\right] \nonumber
\end{eqnarray}
and
\[
R_{n}(x, y)=
\frac{1}{(n+1)!}\sum_{j=0}^{n+1}
\left(
\begin{array}{c}
n+1 \\
 j
\end{array}
\right)
(x-x_{0})^{n+1-j}(y-y_{0})^{j}
\frac{\partial^{n+1} f}{\partial x^{n+1-j}\partial y^{j}}(\xi, \mu) .
\]
Here
\[
\left(
\begin{array}{c}
n \\
 j
\end{array}
\right)=\frac{n!}{j!(n-j)!}
\]
are binomial coefficients.
$T_{n}(x, y)$ is called the {\it n-th Taylor polynomial in
two variables} and $R_{n}(x, y)$ is the remainder term.
\end{theorem}
