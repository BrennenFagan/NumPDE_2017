\documentclass[10pt]{article}

\usepackage{amsmath,amssymb,amsfonts,amsbsy}
\usepackage{epsfig,array}


\textwidth 17cm \textheight 24cm \setlength{\oddsidemargin}{-3mm}
\setlength{\evensidemargin}{-3mm}
\setlength{\headheight}{-1\baselineskip}
\setlength{\headsep}{-1\baselineskip}

\renewcommand{\section}{\subsection}

\pagestyle{empty}


% Definitions
%================================================================
\def\curl{{\rm curl}\,}
\def\div{{\rm div}\,}
\def\la{\lambda}
\def\ba{{\bf a}}
\def\bb{{\bf b}}
\def\be{{\bf e}}
\def\bj{{\bf j}}
\def\bn{{\bf n}}
\def\bV{{\bf V}}
\def\bJ{{\bf J}}
\def\bv{{\bf v}}
\def\bu{{\bf u}}
\def\bH{{\bf H}}
\def\vf{{\bf f}}
\def\bh{{\bf h}}
\def\bU{{\bf U}}
\def\bV{{\bf V}}
\def\bx{{\bf x}}
\def\by{{\bf y}}
\def\bz{{\bf z}}
\def\bX{{\bf X}}
\def\bC{{\bf C}}

\def\vx {{\bf x}}
\def\vy {{\bf y}}
\def\vz {{\bf z}}
\def\vv {{\bf v}}
\def\vu {{\bf u}}
\def\vb {{\bf b}}
\def\vc {{\bf c}}
\def\vr {{\bf r}}

\def\pr{{\partial}}
\def\veta{\boldsymbol{\eta}}
\def\bxi{\boldsymbol{\xi}}
\def\bnu{\boldsymbol{\nu}}
\def\Bom{\boldsymbol{\Omega}}
\def\pd#1#2{\frac{\displaystyle\partial#1}{\displaystyle\partial#2}}
\def\bfr#1#2{\frac{\displaystyle #1}{\displaystyle #2}}
\def\vec#1{\boldsymbol{#1}}
\def\Bbb{\mathbb}
\def \shalf{{\textstyle \frac{1}{2}}}
\def \half{\frac{1}{2}}
\def \ssum{{\textstyle \sum}}


%=======================================================================



\begin{document}


\begin{center}
{{\bf Numerical Methods for PDEs (Spring 2017)}}
\end{center}


\begin{center}
{\large{\bf Solutions 2}}
\end{center}


\centerline{}


\noindent
Consider the heat equation
\begin{equation}
u_{t}-K u_{xx}=0 \quad \hbox{for} \quad 0<x<1, \ \ t>0, \label{1}
\end{equation}
subject to the boundary conditions
\begin{equation}
u(0,t)=0, \quad u(1, t)=0,  \label{2}
\end{equation}
and the initial condition
\begin{equation}
u(x,0)=u_{0}(x).  \label{3}
\end{equation}



%%%%%%%%%%%%%%%%%%%%%%%%%%%%%%%%%%%%%%%%%%%%% Q1 %%%%%%%%%%%%%%%%%%%%%%%%%%%%%%%%%%%%%%%%%%%%%%%%%%%%%%%%%%%



\vskip 0.5cm \noindent
{\bf Problem 4.} Show that the Du Fort -
Frankel method for Eq. (\ref{1}), given by
\[
\frac{w_{k,j+1}-w_{k,j-1}}{2\tau}-K \frac{w_{k+1,
j}-w_{k,j-1}-w_{k,j+1}+w_{k-1,j}}{h^{2}}=0,
\]
has the local truncation error $O\left(\tau^2+h^2+\tau^2/h^2\right)$.

\vskip 0.5cm \noindent
{\bf Solution.} The local truncation error is given by
\begin{equation}
\tau_{kj}=\frac{u_{k,j+1}-u_{k,j-1}}{2\tau}-K \frac{u_{k+1,j}
-u_{k,j-1}-u_{k,j+1}+u_{k-1,j}}{h^{2}}  \label{s1}
\end{equation}
where $u_{kj}=u(x_{k},t_{j})$. Assuming that the
solution $u(x,t)$ is smooth enough, we expand $u_{k\pm 1,j}$ and
$u_{k,j\pm 1}$ in Taylor's series at point $(x_{k},t_{j})$:
\begin{eqnarray}
&&u_{k\pm 1,j}=u(x_{k\pm 1},t_{j})=u(x_{k},t_{j})\pm h\frac{\pr
u}{\pr x}(x_{k},t_{j}) + \frac{h^{2}}{2}\frac{\pr^{2} u}{\pr
x^{2}}(x_{k},t_{j})\pm
\frac{h^{3}}{6}\frac{\pr^{3} u}{\pr x^{3}}(x_{k},t_{j}) +O(h^{4}), \nonumber \\
&&u_{k,j\pm 1}=u(x_{k},t_{j\pm 1})=u(x_{k},t_{j})\pm
\tau\frac{\pr u}{\pr t}(x_{k},t_{j}) + \frac{\tau^{2}}{2}\frac{\pr^{2}
u}{\pr t^{2}}(x_{k},t_{j})\pm \frac{\tau^{3}}{6}\frac{\pr^{3} u}{\pr
t^{3}}(x_{k},t_{j}) +O(\tau^{4}). \nonumber
\end{eqnarray}
It follows that
\begin{eqnarray}
&&u_{k+1,j}+u_{k-1,j}=2u(x_{k},t_{j})+h^2\frac{\pr^{2} u}{\pr
x^{2}}(x_{k},t_{j}) +O(h^{4}), \nonumber \\
&&u_{k,j+1}-u_{k,j-1}=2\tau\frac{\pr u}{\pr t}(x_{k},t_{j})
+\frac{\tau^{3}}{3}\frac{\pr^{3} u}{\pr t^{3}}(x_{k},t_{j}) +O(\tau^{5}), \nonumber \\
&&u_{k,j+1}+u_{k,j-1}=2u(x_{k},t_{j})+ \tau^{2}\frac{\pr^{2}
u}{\pr t^{2}}(x_{k},t_{j})+O(\tau^{4}). \nonumber
\end{eqnarray}
Substituting these in (\ref{1}), we find that
\begin{eqnarray}
\tau_{kj}&=&\frac{\pr u}{\pr t}(x_{k},t_{j})
+\frac{\tau^{2}}{6}\frac{\pr^{3} u}{\pr t^{3}}(x_{k},t_{j})
 +O(\tau^{4})  \nonumber \\
&&-\frac{K}{h^2}\left[2u(x_{k},t_{j})+h^2\frac{\pr^{2} u}{\pr
x^{2}}(x_{k},t_{j}) +O(h^{4})-
2u(x_{k},t_{j})- \tau^{2}\frac{\pr^{2}
u}{\pr t^{2}}(x_{k},t_{j})+O(\tau^{4})\right], \nonumber \\
&=&\frac{\pr u}{\pr t}(x_{k},t_{j})-K\frac{\pr^{2} u}{\pr
x^{2}}(x_{k},t_{j})+\frac{\tau^{2}}{6}\frac{\pr^{3} u}{\pr t^{3}}(x_{k},t_{j})
 +O(\tau^{4})+K\frac{\tau^{2}}{h^2}\frac{\pr^{2}
u}{\pr t^{2}}(x_{k},t_{j})+O(h^{2}) +O(\tau^{4})+O(\tau^4/h^2), \nonumber \\
&=&O(\tau^2+h^2+\tau^2/h^2). \nonumber
\end{eqnarray}


%%%%%%%%%%%%%%%%%%%%%%%%%%%%%%%%%%%%%%%%%%%%% Q2 %%%%%%%%%%%%%%%%%%%%%%%%%%%%%%%%%%%%%%%%%%%%%%%%%%%%%%%%%%%

\vskip 0.5cm \noindent
{\bf Problem 5.} The initial boundary value
problem (\ref{1})--(\ref{3}) is solved numerically using the
finite-difference method:
\begin{eqnarray}
&&w_{k0}=u_{0}(x_{k}), \quad w_{0j}=0, \quad w_{Nj}=0,   \nonumber \\
&&\frac{w_{k,j+1}-w_{k,j}}{\tau}-K(1-\sigma) \frac{w_{k+1,
j}-2w_{kj}+w_{k-1,j}}{h^{2}}-K\sigma \frac{w_{k+1,
j+1}-2w_{k,j+1}+w_{k-1,j+1}}{h^{2}}=0,   \label{4}
\end{eqnarray}
for $k=1, 2, \dots , N-1$ and $j=0, 1, \dots$. Here $w_{kj}$ is an
approximation to $u(x_{k}, y_{j})$ and $x_{k}=k h$
$(k=0,1,\dots,N)$, $t_{j}=j \tau$ $(j=0,1,\dots)$, $h=\frac{1}{N}$.
In Eqs. (\ref{4}), $\sigma$ is a real parameter such that
$0\leq \sigma\leq 1$. Show that the method is stable if
\[
\sigma\geq\frac{1}{2}\left(1-\frac{1}{2\gamma}\right)
\]
where $\gamma=K\tau/h^2$.

\vskip 0.5cm \noindent
{\bf Solution.}
The perturbation  $z_{kj}$ satisfies the difference equation
\begin{equation}
\frac{z_{k,j+1}-z_{k,j}}{\tau}-K(1-\sigma) \frac{z_{k+1,
j}-2z_{kj}+z_{k-1,j}}{h^{2}}-K\sigma \frac{z_{k+1,
j+1}-2z_{k,j+1}+z_{k-1,j+1}}{h^{2}}=0 \label{s2}
\end{equation}
for $k=1,2, \dots, N-1$ and $j=1,2, \dots$. We will seek a particular solution of
(\ref{s2}) in the form
\begin{equation}
z_{k,j}=\rho_{q}^{j}e^{iqx_{k}}, \quad q\in{\mathbb R}. \label{s3}
\end{equation}
The finite-difference method (\ref{4}) is stable with respect to
initial condition, if $\vert\rho_{q}\vert\leq 1$ for all $q\in{\mathbb R}$.
Substitution of (\ref{s3}) into (\ref{4}) yields
\[
\frac{e^{iqx_{k}}}{\tau}\left(\rho_{q}^{j+1}-\rho^{j}\right)-
K(1-\sigma)\frac{\rho_{q}^{j}}{h^{2}}
\left(e^{iqx_{k+1}}-2e^{iqx_{k}}+e^{iqx_{k-1}}\right)-
K\sigma\frac{\rho_{q}^{j+1}}{h^{2}}
\left(e^{iqx_{k+1}}-2e^{iqx_{k}}+e^{iqx_{k-1}}\right)=0
\]
or
\[
\rho_{q}-1-
\gamma(1-\sigma)\left(e^{iqh}-2+e^{-iqh}\right)-
\rho_{q}\gamma\sigma\left(e^{iqh}-2+e^{-iqh}\right)=0.
\]
Since
\[
e^{iqh}-2+e^{-iqh}=\left(e^{iqh/2}-e^{-iqh/2}\right)^{2}=-4\sin^{2} \frac{qh}{2},
\]
we obtain
\[
\rho_{q}=\frac{1-4\gamma(1-\sigma)\sin^{2} \frac{qh}{2}}
{1+4\gamma\sigma\sin^{2} \frac{qh}{2}}.
\]
Further, we have
\[
\vert\rho_{q}\vert\leq 1 \quad \Rightarrow \quad -1\leq\rho_{q}
\quad \Rightarrow \quad 4\gamma(1-2\sigma)\sin^{2} \frac{qh}{2}\leq 2.
\]
The last inequality holds for all $q$, provided that
\[
4\gamma(1-2\sigma)\leq 2
\quad {\rm or} \quad \sigma\geq\frac{1}{2}\left(1-\frac{1}{2\gamma}\right),
\]
which is the required stability  condition.


%%%%%%%%%%%%%%%%%%%%%%%%%%%%%%%%%%%%%%%%%%%%% Q3 %%%%%%%%%%%%%%%%%%%%%%%%%%%%%%%%%%%%%%%%%%%%%%%%%%%%%%%%%%%

\vskip 0.5cm
\noindent
{\bf Problem 6.} Show that if
\[
\gamma\equiv\frac{K\tau}{h^2}=\frac{1}{6}
\]
in the explicit forward-difference method for Eq. (\ref{1}):
\[
\frac{w_{k,j+1}-w_{kj}}{\tau}-K
\frac{w_{k+1, j}-2w_{kj}+w_{k-1,j}}{h^{2}}=0,
\]
then the local truncation error is $O(\tau^2)$ or, equivalently $O(h^4)$.


\vskip 0.5cm \noindent
{\bf Solution.} The local truncation error is given by
\begin{equation}
\tau_{kj}=\frac{u_{k,j+1}-u_{kj}}{\tau}-K \frac{u_{k+1,
j}-2u_{kj}+u_{k-1,j}}{h^{2}}  \label{s5}
\end{equation}
where $u_{kj}=u(x_{k},t_{j})$. Assuming that the
solution $u(x,t)$ is smooth enough, we expand $u_{k\pm 1,j}$ and
$u_{k,j+1}$ in Taylor's series at point $(x_{k},t_{j})$:
\begin{eqnarray}
&&u_{k\pm 1,j}=u(x_{k\pm 1},t_{j})=u(x_{k},t_{j})\pm h\frac{\pr
u}{\pr x}(x_{k},t_{j}) + \frac{h^{2}}{2}\frac{\pr^{2} u}{\pr
x^{2}}(x_{k},t_{j})\pm
\frac{h^{3}}{6}\frac{\pr^{3} u}{\pr x^{3}}(x_{k},t_{j})
+ \frac{h^{4}}{24}\frac{\pr^{2} u}{\pr x^{4}}(x_{k},t_{j})+O(h^{5}), \nonumber \\
&&u_{k,j+ 1}=u(x_{k},t_{j+1})=u(x_{k},t_{j})+
\tau\frac{\pr u}{\pr t}(x_{k},t_{j}) + \frac{\tau^{2}}{2}\frac{\pr^{2}
u}{\pr t^{2}}(x_{k},t_{j})+ \frac{\tau^{3}}{6}\frac{\pr^{3} u}{\pr
t^{3}}(x_{k},t_{j}) +O(\tau^{4}). \nonumber
\end{eqnarray}
It follows that
\begin{eqnarray}
&&\frac{u_{k,j+1}-u_{kj}}{\tau}=
\frac{\pr u}{\pr t}(x_{k},t_{j}) + \frac{\tau}{2}\frac{\pr^{2}
u}{\pr t^{2}}(x_{k},t_{j})+ O(\tau^{2}), \nonumber \\
&&\frac{u_{k+1, j}-2u_{kj}+u_{k-1,j}}{h^{2}}=
\frac{\pr^{2} u}{\pr x^{2}}(x_{k},t_{j})
+ \frac{h^{2}}{12}\frac{\pr^{2} u}{\pr x^{4}}(x_{k},t_{j})+O(h^{4}). \nonumber
\end{eqnarray}
Hence,
\begin{eqnarray}
\tau_{kj}&=&
\frac{\pr u}{\pr t}(x_{k},t_{j})-K\frac{\pr^{2} u}{\pr x^{2}}(x_{k},t_{j})
+ \frac{\tau}{2}\frac{\pr^{2}u}{\pr t^{2}}(x_{k},t_{j})
-K \frac{h^{2}}{12}\frac{\pr^{2} u}{\pr x^{4}}(x_{k},t_{j})
+ O(\tau^{2})+O(h^{4})  \nonumber \\
&=&\frac{\tau}{2}\frac{\pr^{2}u}{\pr t^{2}}(x_{k},t_{j})
-K \frac{h^{2}}{12}\frac{\pr^{2} u}{\pr x^{4}}(x_{k},t_{j})
+ O(\tau^{2})+O(h^{4}).
\end{eqnarray}
From Eq. (\ref{1}), we have
\[
\frac{\pr^{2}u}{\pr t^{2}}=K\frac{\pr^{2} }{\pr x^{2}}
\frac{\pr u}{\pr t}=K^2\frac{\pr^{4} u}{\pr x^{4}}.
\]
Therefore, we can rewrite the above formula for $\tau_{kj}$ in the form
\[
\tau_{kj}=
K\frac{h^{2}}{2}\left(\frac{\tau K}{h^2}-\frac{1}{6}\right)
\frac{\pr^{2} u}{\pr x^{4}}(x_{k},t_{j})
+ O(\tau^{2})+O(h^{4}).
\]
It follows that if
\[
\frac{K\tau}{h^2}=\frac{1}{6},
\]
then $\tau_{kj}=O(\tau^{2})+O(h^{4})$ or, in view of the last formula,
\[
\tau_{kj}=O(\tau^{2})=O(h^{4}).
\]


 %%%%%%%%%%%%%%%%%%%%%%%%%%%%%%%%%%%%%%%%%%%%%%%%%%%%%%%%%%%




\vskip 0.5cm
\noindent
{\bf Problem 7.}
Devise a backward difference scheme of $O(\tau+h^2)$ for the non-homogeneous heat
equation eq.(2.73) in the notes with boundary conditions as given in eq.(2.80). This means
you need to derive an expression for $w_{0,j+1}$ similar to eq.(2.78) and also a similar
expression for $w_{N,j+1}$.


\vskip 0.5cm \noindent {\bf Solution.}
The backward difference formula at the left endpoint $x_0=0$ is
\[
\frac{w_{0j}-w_{0,j-1}}{\tau}-K\frac{w_{1j}-2w_{0j}+w_{-1,j}}{h^2}=f(0,t_j).
\]
Using the central difference formula to approximate the derivative term in the boundary condition
\[
\frac{\pr u}{\pr x}(0,t)+c_{1}(t)u(0,t)=\mu_{1}(t)
\]
gives
\[w_{-1,j}=w_{1j}+2hc_1(t_j)w_{0j}-2h\mu_1(t_j).
\]
Substituting this into the backward difference formula and multiplying by $\tau$ gives
\[
w_{0j}-w_{0,j-1}-\gamma\left(w_{1j}-2w_{0j}+w_{1j}+2hc_1(t_j)w_{0j}-2h\mu_1(t_j)\right)=\tau f(0,t_j).
\]
Collecting terms gives
\[
w_{0j}\left(1+2\gamma(1-hc_1(t_j))\right)=2\gamma w_{1j}+w_{0,j-1}+\tau f(0,t_j)-2h\gamma\mu_1(t_j).
\]
Thus
\[
w_{0j} = \frac{2\gamma}{1+2\gamma(1-hc_1(t_j))} w_{1j}
+\frac{w_{0,j-1}+\tau f(0,t_j)-2h\gamma\mu_1(t_j)}{1+2\gamma(1-hc_1(t_j))}.
\]
Similarly for the right endpoint $x_N=L$ we have the backward difference formula
\[
\frac{w_{N,j}-w_{N,j-1}}{\tau}-K\frac{w_{N+1,j}-2w_{N,j}+w_{N-1,j}}{h^2}=f(L,t_j).
\]
Using the central difference formula to approximate the derivative term in the boundary condition
\[
\frac{\pr u}{\pr x}(L,t)+c_{2}(t)u(L,t)=\mu_{2}(t)
\]
gives
\[w_{N+1,j}=w_{N-1,j}-2hc_2(t_j)w_{N,j}+2h\mu_2(t_j).
\]
Substituting this into the backward difference formula and multiplying by $\tau$ gives
\[
w_{N,j}-w_{N,j-1}-\gamma\left(w_{N-1,j}-2hc_2(t_j)w_{N,j}+2h\mu_2(t_j)-2w_{N,j}+w_{N-1,j}\right)=\tau f(L,t_j).
\]
Collecting terms gives
\[
w_{N,j}\left(1+2\gamma(1+hc_2(t_j))\right)=2\gamma w_{N-1,j}+w_{N,j-1}+\tau f(L,t_j)+2h\gamma\mu_2(t_j).
\]
Thus
\[
w_{N,j} = \frac{2\gamma}{1+2\gamma(1+hc_2(t_j))} w_{N-1,j}
+\frac{w_{N,j-1}+\tau f(L,t_j)+2h\gamma\mu_2(t_j)}{1+2\gamma(1+hc_2(t_j))}.
\]


 %%%%%%%%%%%%%%%%%%%%%%%%%%%%%%%%%%%%%%%%%%%%%%%%%%%%%%%%%%%

\vskip 0.5cm
\noindent
{\bf Problem 8.}
Consider the equation
\[
\frac{\partial u}{\partial t}=3\frac{\partial^{2}u}{\partial x^{2}}-
2a(x,t)\frac{\partial u}{\partial x}+b(x,t) \quad \hbox{for} \quad
0<x<1, \ \ t > 0,
\]
subject to the initial and boundary conditions
\[
u(0,t)=\mu_{1}(t), \quad u(1, t)=\mu_{2}(t),
\quad u(x, 0)=u_{0}(x) .
\]
Obtain a finite-difference approximation to this boundary-value
problem and show that your finite-difference method is consistent
with the equation, i.e. that the local truncation errors tend to
zero as step sizes in $x$ and in $t$ go to zero.



\vskip 0.5cm \noindent {\bf Solution.} Let $(x_{k},t_{j})$ be the
grid points, where $x_{k}=hk$ ($k=0,1,\dots,N$), $t_{j}=\tau j$
($k=0,1,\dots$), $h=1/N$. Employing the central difference
formulae for the first and second order derivatives with respect
to $x$ and the two-point forward difference formula for $\pr u/\pr
t$ at point $(x_{k},t_{j})$, we obtain the following
finite-difference equation
\[
\frac{w_{k,j+1}-w_{k,j}}{\tau}=3\frac{w_{k+1,j}-2w_{k,j}+w_{k-1,j}}{h^{2}}-
2a(x_{k},t_{j})\frac{w_{k+1,j}-w_{k-1,j}}{2h}+b(x_{k},t_{j}),
\]
for $k=1,2,\dots,N-1$, $j=0,1,\dots$. From initial and boundary
conditions, we obtain
\[
w_{0,j}=\mu_{1}(t_{j}), \quad w_{N1,j}=\mu_{2}(t_{j}),
\quad w_{k,0}=u_{0}(x_{k}) .
\]

\vskip 0.3cm \noindent
The local truncation error in given by
\[
\tau_{kj}=\frac{u_{k,j+1}-u_{k,j}}{\tau}-3\frac{u_{k+1,j}-2u_{k,j}+u_{k+1,j}}{h^{2}}+
2a(x_{k},t_{j})\frac{u_{k+1,j}-u_{k-1,j}}{2h}-b(x_{k},t_{j}),
\]
where $u_{kj}=u(x_{k},t_{j})$. Since (see Solution to Problem 6 above)
\[
\frac{u_{k,j+1}-u_{kj}}{\tau}=
\frac{\pr u}{\pr t}(x_{k},t_{j}) + O(\tau), \quad
\frac{u_{k+1, j}-2u_{kj}+u_{k-1,j}}{h^{2}}=
\frac{\pr^{2} u}{\pr x^{2}}(x_{k},t_{j})
+ O(h^{2})
\]
and since
\[
\frac{u_{k+1,j}-u_{k-1,j}}{2h}=\frac{\pr u}{\pr x}(x_{k},t_{j})
+O(h^{2}),
\]
We obtain
\[
\tau_{kj}=
\frac{\pr u}{\pr t}(x_{k},t_{j})-3\frac{\pr^{2} u}{\pr x^{2}}(x_{k},t_{j})+
2a(x_{k},t_{j})\frac{\pr u}{\pr x}(x_{k},t_{j})-b(x_{k},t_{j})
+O(\tau)+O(h^2)=O(\tau+h^2).
\]
Hence,
\[
\tau_{kj} \to 0 \quad {\rm as} \ \ \tau\to 0, \ \ h\to 0,
\]
so that the method is consistent.

\vskip 0.3cm \noindent
{\bf Remark.} The method described above is not the only possible finite-difference
method to solve the given problem.


\end{document}
