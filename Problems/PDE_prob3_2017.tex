\documentclass[10pt]{article}

\usepackage{amsmath,amssymb,amsfonts,amsbsy}
\usepackage{epsfig,array}


\textwidth 17cm \textheight 24cm \setlength{\oddsidemargin}{-3mm}
\setlength{\evensidemargin}{-3mm}
\setlength{\headheight}{-1\baselineskip}
\setlength{\headsep}{-1\baselineskip}

\renewcommand{\section}{\subsection}

\pagestyle{empty}



% Definitions
%================================================================
\def\curl{{\rm curl}\,}
\def\div{{\rm div}\,}
\def\la{\lambda}
\def\ba{{\bf a}}
\def\bb{{\bf b}}
\def\be{{\bf e}}
\def\bj{{\bf j}}
\def\bn{{\bf n}}
\def\bV{{\bf V}}
\def\bJ{{\bf J}}
\def\bv{{\bf v}}
\def\bu{{\bf u}}
\def\bH{{\bf H}}
\def\vf{{\bf f}}
\def\bh{{\bf h}}
\def\bU{{\bf U}}
\def\bV{{\bf V}}
\def\bx{{\bf x}}
\def\by{{\bf y}}
\def\bz{{\bf z}}
\def\bX{{\bf X}}
\def\bC{{\bf C}}

\def\vx {{\bf x}}
\def\vy {{\bf y}}
\def\vz {{\bf z}}
\def\vv {{\bf v}}
\def\vu {{\bf u}}
\def\vb {{\bf b}}
\def\vc {{\bf c}}
\def\vr {{\bf r}}

\def\pr{{\partial}}
\def\veta{\boldsymbol{\eta}}
\def\bxi{\boldsymbol{\xi}}
\def\bnu{\boldsymbol{\nu}}
\def\Bom{\boldsymbol{\Omega}}
\def\pd#1#2{\frac{\displaystyle\partial#1}{\displaystyle\partial#2}}
\def\bfr#1#2{\frac{\displaystyle #1}{\displaystyle #2}}
\def\vec#1{\boldsymbol{#1}}
\def\Bbb{\mathbb}
\def \shalf{{\textstyle \frac{1}{2}}}
\def \half{\frac{1}{2}}
\def \ssum{{\textstyle \sum}}

%=======================================================================

\begin{document}

\begin{center}
{\large{\bf Numerical Methods for PDEs (Spring 2017)}}
\end{center}

\begin{center}
{\large{\bf Problems 3}}
\end{center}

\noindent
{\bf Hand in your written solution to problem 11 at the start of the lecture on 7 March. Show me your code for that problem in the practical on 10 March.}

\vskip 0.5cm \noindent
{\bf Problem 9.} By  expanding $g(x\pm h)$, $Q(x\pm h)$ in Taylor's series at $x$,
show that
\[
\frac{d }{d x}\left(Q(x)
\frac{d g}{\pr x}\right)=\frac{1}{h^2}
\left(Q_{+}\left[g(x+h)-g(x)\right]-
Q_{-}\left[g(x)-g(x-h)\right]\right) +O(h^2),
\]
where
\[
Q_{\pm}=\frac{1}{2}\left[Q(x)+Q(x\pm h)\right].
\]

\vskip 0.5cm \noindent
{\bf Problem 10.} Consider the two-dimensional heat equation
\begin{equation}
\frac{\partial u}{\partial t}-K \left(\frac{\partial^{2}u}{\partial
x^{2}}+\frac{\partial^{2}u}{\partial
y^{2}}\right)=0 \quad \hbox{for} \quad 0<x<L_{1}, \ \ y<x<L_{2}, \ \ t>0, \label{1}
\end{equation}
subject to the boundary conditions
\[
u(0,y,t)=0, \quad u(L_{1},y,t)=0, \quad
u(x,0,t)=0, \quad u(x,L_{2},t)=0,
\]
and the initial condition
\[
u(x,y,0)=u_{0}(x,y).
\]
At interior grid points $(x_{x},y_{j},t_{n})$, equation (\ref{1}) is approximated by the finite-difference scheme
\[
\frac{w_{kj}^{n}-w_{kj}^{n-1}}{\tau} -K\left(\frac{\delta_{x}^2}{h_{1}^2}
+\frac{\delta_{y}^2}{h_{2}^2}\right)w_{kj}^{n}=0,
\]
where $w_{kj}^{n}$ are approximations to $u(x_{x},y_{j},t_{n})$; $x_{k}=k h_{1}$ for $k=0,1,\dots N_{1}$,
$h_{1}=L_{1}/N_{1}$; $y_{j}=j h_{2}$ for $j=0,1,\dots N_{2}$,
$h_{2}=L_{2}/N_{2}$; $t_{n}=n\tau$ for $n=0,1,\dots$ and $\tau$ in the length of the time step.

\vskip 0.3cm
\noindent
Investigate the stability of this scheme by the Fourier method.

\vskip 0.5cm \noindent
{\bf Problem 11.} The nonlinear heat equation
\[
u_t -K u_{xx}=f(u) \quad \textrm{for} \quad 0 < x < L, \ \ 0 < t < T
\]
(where $K$ is a constant and $f(u)$ is a given function), subject to the initial and boundary conditions
\[
u(x,0)=u_0(x) \ \ \hbox{for} \ 0 < x < L, \quad u(0,t)=u(L,t)=0  \ \ \hbox{for} \ 0 < t < T,
\]
is solved using the finite-difference method:
\begin{eqnarray}
&&\frac{w_{k,j}-w_{k,j-1}}{\tau}-
K \, \frac{w_{k+1,j}-2w_{k,j}+w_{k-1,j}}{h^{2}}=f(w_{kj}) \ \ \hbox{for} \ \ k=1,2,\dots,N-1 \  \hbox{and} \ j=1,2,\dots,M; \label{pp11} \\
&&w_{k,0}=u_0(x_{k}) \ \ \hbox{for} \ k=0, \dots,N \ \ \hbox{and} \ \ w_{0,j}=w_{N,j}=0 \ \ \hbox{for} \ j=1, \dots,M . \label{pp22}
\end{eqnarray}
Obtain the computation formulae for solving the nonlinear equations (\ref{pp11}) by the Newton method and implement the solution method in R.



\end{document}
