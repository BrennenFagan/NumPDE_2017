\documentclass[10pt]{article}

\usepackage{amsmath,amssymb,amsfonts,amsbsy}
\usepackage{epsfig,array}


\textwidth 17cm \textheight 24cm \setlength{\oddsidemargin}{-3mm}
\setlength{\evensidemargin}{-3mm}
\setlength{\headheight}{-1\baselineskip}
\setlength{\headsep}{-1\baselineskip}

\renewcommand{\section}{\subsection}

\pagestyle{empty}



% Definitions
%================================================================
\def\curl{{\rm curl}\,}
\def\div{{\rm div}\,}
\def\la{\lambda}
\def\ba{{\bf a}}
\def\bb{{\bf b}}
\def\be{{\bf e}}
\def\bj{{\bf j}}
\def\bn{{\bf n}}
\def\bV{{\bf V}}
\def\bJ{{\bf J}}
\def\bv{{\bf v}}
\def\bu{{\bf u}}
\def\bH{{\bf H}}
\def\vf{{\bf f}}
\def\bh{{\bf h}}
\def\bU{{\bf U}}
\def\bV{{\bf V}}
\def\bx{{\bf x}}
\def\by{{\bf y}}
\def\bz{{\bf z}}
\def\bX{{\bf X}}
\def\bC{{\bf C}}

\def\vx {{\bf x}}
\def\vy {{\bf y}}
\def\vz {{\bf z}}
\def\vv {{\bf v}}
\def\vu {{\bf u}}
\def\vb {{\bf b}}
\def\vc {{\bf c}}
\def\vr {{\bf r}}

\def\pr{{\partial}}
\def\veta{\boldsymbol{\eta}}
\def\bxi{\boldsymbol{\xi}}
\def\bnu{\boldsymbol{\nu}}
\def\Bom{\boldsymbol{\Omega}}
\def\pd#1#2{\frac{\displaystyle\partial#1}{\displaystyle\partial#2}}
\def\bfr#1#2{\frac{\displaystyle #1}{\displaystyle #2}}
\def\vec#1{\boldsymbol{#1}}
\def\Bbb{\mathbb}
\def \shalf{{\textstyle \frac{1}{2}}}
\def \half{\frac{1}{2}}
\def \ssum{{\textstyle \sum}}

%=======================================================================

\begin{document}

\begin{center}
{\large{\bf Numerical Methods for PDEs (Spring 2017)}}
\end{center}

\begin{center}
{\large{\bf Problems 4}}
\end{center}

\noindent
{\bf Hand in your written solution to problem 14 at the start of the lecture on 14 March.}

\vskip 0.5cm \noindent {\bf Problem 12.} Consider the following equation
\begin{equation}
u_{xx}+(x^2+y^2)u_{yy}+2u_{x}= f(x,y) \quad \hbox{for} \quad
0<x<1, \ \ 0< y < 1, \label{4}
\end{equation}
subject to the boundary conditions
\begin{equation}
u(0,y)=u(1, y)=0, \quad u(x,0)=u(x,1)=0. \label{5}
\end{equation}
(a) Obtain a finite-difference approximation to this boundary-value
problem and show that your finite-difference method is consistent with the equation,
i.e. that the local truncation errors tend to zero as step sizes in $x$ and in $y$
go to zero.

\vskip 0.3cm \noindent
(b) Modify your finite-difference method for the case of the following boundary conditions:
\begin{equation}
u_{x}(0,y)=u_{x}(1, y)=0, \quad u(x,0)=u(x,1)=0. \label{5b}
\end{equation}



\vskip 0.5cm \noindent
{\bf Problem 13.} Consider the boundary value problem
\begin{eqnarray}
&&u_{xx}+ u_{yy} = 0 \quad
{\rm for} \quad 0< x< 1, \ \ 0< y< 1;  \nonumber \\
&&u(0,y)=y, \quad u(1,y)=1-y \quad {\rm for} \quad 0< y< 1;  \nonumber \\
&&u(x,0)=x, \quad u(x,1)=1-x \quad {\rm for} \quad 0< x< 1.  \label{6}
\end{eqnarray}
For the grid $(x_{k},y_{j})=(kh,jh)$ for $k,j=0,1,2,3$, with $h=1/3$,
write down the system of linear equations for $w_{1,1}$,
$w_{1,2}$, $w_{2,1}$, $w_{2,2}$ [where $w_{k,j}\approx u(x_{k},t_{j})$],
solve it and compare the result with the exact solution given by
\[
u=-2xy+x+y.
\]

\vskip 5mm
\noindent
{\bf Problem 14.} Consider the following
elliptic equation
\begin{equation}
u_{xx}+u_{xy}+u_{yy}=
f(x,y) \quad \hbox{for} \quad 0<x<1, \ \ 0< y < 1, \label{7}
\end{equation}
subject to the boundary conditions
\begin{equation}
u(0,y)=u(1, y)=0, \quad u(x,0)=u(x,1)=0. \label{8}
\end{equation}
Consider the square grid with step size $h=1/N$ in both $x$ and $y$ directions and
obtain a finite-difference approximation to this boundary-value
problem with truncation error $O(h^2)$ (calculate the truncation error of your scheme explicitly using Taylor expansions).



\end{document}



\end{document}
